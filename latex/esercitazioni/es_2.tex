\documentclass[11pt,a4]{article}

\usepackage[margin=2cm]{geometry}


\usepackage{collectbox}

\newcommand{\mybox}[2]{$\quad$\fbox{
\begin{minipage}{#1cm}
\hfill\vspace{#2cm}
\end{minipage}
}}

\usepackage{fancyhdr}
\pagestyle{fancy}
\rhead{Programmazione 1 - Esercitazione 2}

\usepackage[T1]{fontenc}
\usepackage[utf8]{inputenc}
\usepackage{lmodern}
%%%%%%%%%%%%%%%%%%%%%%%%%%%%%%%%%%%%%%%%%%%%%%%%%%%%%%%%%
% Source: http://en.wikibooks.org/wiki/LaTeX/Hyperlinks %
%%%%%%%%%%%%%%%%%%%%%%%%%%%%%%%%%%%%%%%%%%%%%%%%%%%%%%%%%
\usepackage{hyperref}
\usepackage{graphicx}
\usepackage[english]{babel}

\usepackage{bm}
\usepackage{amsmath}
\usepackage{amsfonts}

\usepackage{amsthm}
\newtheorem{definition}{Definizione}
\newtheorem{theorem}{Teorema}
\renewcommand*{\proofname}{Dimostrazione}
\newtheorem{example}{Esempio}
\newtheorem{lemma}{Lemma}
\newtheorem{exercise}{Esercizio}
\newtheorem{property}{Proprietà}

\usepackage[ruled,vlined,linesnumbered]{algorithm2e}

\newcommand{\xstar}{x^*}
\newcommand{\bxstar}{\bm{x^*}}
\newcommand{\bx}{\bm{x}}
\newcommand{\Rn}{\mathbb{R}^n}
\newcommand{\RR}{\mathbb{R}}
\newcommand{\norm}[1]{\left\lvert \left\lvert #1 \right\lvert \right\lvert}

\newcommand{\fx}{f(x)}

\newcommand{\gradfx}{\nabla \fx}
\newcommand{\Gx}{\nabla f(x)}
\newcommand{\Gk}{\nabla f(x_k)}
\newcommand{\Gs}{\nabla f(\xstar)}

\newcommand{\Hx}{\nabla^2 f(x)}
\newcommand{\Hk}{\nabla^2 f(x_k)}
\newcommand{\Hs}{\nabla^2 f(\xstar)}
\newcommand{\hess}{\nabla^2 f}

\newcommand{\step}{\alpha}
\newcommand{\Seqx}{\{ x_k \}}

\usepackage{mathtools}
\newcommand\myeq{\stackrel{\mathclap{\normalfont\mbox{def}}}{=}}

\usepackage{listings}
\lstset
{ 
    language=Matlab,
    basicstyle=\normalsize,
    numbers=left,
    stepnumber=1,
    showstringspaces=false,
    tabsize=1,
    breaklines=true,
    breakatwhitespace=false,
   frame=single
}


\begin{document}
\thispagestyle{empty}
\hrule
\begin{center}
   {\Large {\bf Programmazione 1 \hspace{3cm} $\quad \quad \quad$ Esercitazione 2}}
\end{center}
{\bf Cognome: }\hspace{2.5cm} {\bf Nome: } \hspace{2.5cm} {\bf Matricola: } \\\
\hrule

\begin{enumerate}
\section*{}

%%%%%%%%%%%%%%%%%%%%%%%%%%%%%%%%%%%%%%%%%%%%%%%%%%%%%%%%%%%%%%%%%%%%%%%%%%%%%
\item Scrivere una procedura che calcoli l'ennesimo numero di Fibonacci usando un {\bf processo iterativo}
(si veda il {\tt Lab 4} per la definizione di processo iterativo).

\mybox{15}{3.5}

%%%%%%%%%%%%%%%%%%%%%%%%%%%%%%%%%%%%%%%%%%%%%%%%%%%%%%%%%%%%%%%%%%%%%%%%%%%%%
\item Una funzione $f$ è definita dalla regola seguente:
\begin{align}
	f(n) = \left\{\begin{array}{ll} n & \mbox{if } n<3 \\ f(n-1)+2\,f(n-2)+3\,f(n-3) & \mbox{if } n \geq 3 \end{array} \right.
\end{align}
\begin{enumerate}
\item Si scriva una procedura che calcoli $f$ usando un {\bf processo ricorsivo}.
\item Si scriva una procedura che calcoli $f$ usando un {\bf processo iterativo}.
\end{enumerate}

\mybox{15}{3.5}

%%%%%%%%%%%%%%%%%%%%%%%%%%%%%%%%%%%%%%%%%%%%%%%%%%%%%%%%%%%%%%%%%%%%%%%%%%%%%
\item Si scrive un predicato che ci dica se un dato numero $n$ sia un numero primo.
Si può usare la definizione che $n$ è un numero primo, se e solo se $n$ è il suo più piccolo divisore maggiore di uno
(suggerimento: scrivere prima una funzione che trova il più piccolo divisore di $n$).

È possibile scrivere questa procedura in modo tale che l'ordine di crescita per il numero di operazioni richieste sia $\Theta(\sqrt{n})$?

\mybox{15}{3.5}
 
%%%%%%%%%%%%%%%%%%%%%%%%%%%%%%%%%%%%%%%%%%%%%%%%%%%%%%%%%%%%%%%%%%%%%%%%%%%%%
\item La procedura {\tt Sommatoria} vista nel {\tt Lab 7} genera un processo ricorsivo lineare.
La stessa procedura può essere riscritta in modo tale che il processo generato sia iterativo:
scrivere la procedura che genera un processo iterativo lineare.

\mybox{15}{4.5}

%%%%%%%%%%%%%%%%%%%%%%%%%%%%%%%%%%%%%%%%%%%%%%%%%%%%%%%%%%%%%%%%%%%%%%%%%%%%%
\item In maniera analoga alla funzione {\tt Sommatoria}, si può definire una funzione {\tt Produttoria}
che restituisce i valori dei prodotti di una funzione valutata in un insieme di punti definiti da un dato intervallo $[a,b]$:

$$
	\prod_{n=a}^b f(n) = f(a)\cdot ... \cdot f(b)
$$

Scrivere tale funzione e mostrare come sia possibile usarla per calcolare $n!$

\mybox{15}{4.5}

%%%%%%%%%%%%%%%%%%%%%%%%%%%%%%%%%%%%%%%%%%%%%%%%%%%%%%%%%%%%%%%%%%%%%%%%%%%%% 
\item Eseguire il grafico sovrapposto delle funzioni seguenti: 
$$R(n)=n, R(n)=10^5\,n, R(n)=n^2, R(n)=\log{n}, R(n)=n\log{n}, R(n)=1.6^n, R(n)=2^n$$

\mybox{15}{4.5}

%%%%%%%%%%%%%%%%%%%%%%%%%%%%%%%%%%%%%%%%%%%%%%%%%%%%%%%%%%%%%%%%%%%%%%%%%%%%%
\item {\bf CHALLENGE (facoltativo)}: Si scriva una procedura che calcoli i numeri di Fibonacci
con un ordine di crescita che sia $\Theta(log(n))$. Mandare la soluzione per email.
\end{enumerate}

\end{document}
