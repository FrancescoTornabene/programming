\documentclass[11pt,a4]{article}

\usepackage[margin=2cm]{geometry}

\usepackage{amsmath}
\usepackage{url}

\usepackage{amsmath}
\usepackage{url}

\usepackage[utf8]{inputenc}

% Default fixed font does not support bold face
\DeclareFixedFont{\ttb}{T1}{txtt}{bx}{n}{10} % for bold
\DeclareFixedFont{\ttm}{T1}{txtt}{m}{n}{10}  % for normal

% Custom colors
\usepackage{color}
\definecolor{deepblue}{rgb}{0,0,0.5}
\definecolor{deepred}{rgb}{0.6,0,0}
\definecolor{deepgreen}{rgb}{0,0.5,0}

\usepackage{listings}

% Python style for highlighting
\newcommand\pythonstyle{\lstset{
language=Python,
basicstyle=\ttm,
otherkeywords={self},             % Add keywords here
keywordstyle=\ttb\color{deepblue},
emph={MyClass,__init__},          % Custom highlighting
emphstyle=\ttb\color{deepred},    % Custom highlighting style
stringstyle=\color{deepgreen},
frame=tb,                         % Any extra options here
showstringspaces=false            % 
}}


% Python environment
\lstnewenvironment{python}[1][]
{
\pythonstyle
\lstset{#1}
}
{}

% Python for external files
\newcommand\pythonexternal[2][]{{
\pythonstyle
\lstinputlisting[#1]{#2}}}

% Python for inline
\newcommand\pythoninline[1]{{\pythonstyle\lstinline!#1!}}


\usepackage{collectbox}

\newcommand{\mybox}[2]{$\quad$\fbox{
\begin{minipage}{#1cm}
\hfill\vspace{#2cm}
\end{minipage}
}}


\usepackage{fancyhdr}
\pagestyle{fancy}
\rhead{Programmazione 1 - Esercitazione 3}

\usepackage[T1]{fontenc}
\usepackage[utf8]{inputenc}
\usepackage{lmodern}
%%%%%%%%%%%%%%%%%%%%%%%%%%%%%%%%%%%%%%%%%%%%%%%%%%%%%%%%%
% Source: http://en.wikibooks.org/wiki/LaTeX/Hyperlinks %
%%%%%%%%%%%%%%%%%%%%%%%%%%%%%%%%%%%%%%%%%%%%%%%%%%%%%%%%%
\usepackage{hyperref}
\usepackage{graphicx}
\usepackage[english]{babel}

\usepackage{bm}
\usepackage{amsmath}
\usepackage{amsfonts}

\usepackage{amsthm}
\newtheorem{definition}{Definizione}
\newtheorem{theorem}{Teorema}
\renewcommand*{\proofname}{Dimostrazione}
\newtheorem{example}{Esempio}
\newtheorem{lemma}{Lemma}
\newtheorem{exercise}{Esercizio}
\newtheorem{property}{Proprietà}

\usepackage[ruled,vlined,linesnumbered]{algorithm2e}

\newcommand{\xstar}{x^*}
\newcommand{\bxstar}{\bm{x^*}}
\newcommand{\bx}{\bm{x}}
\newcommand{\Rn}{\mathbb{R}^n}
\newcommand{\RR}{\mathbb{R}}
\newcommand{\norm}[1]{\left\lvert \left\lvert #1 \right\lvert \right\lvert}

\newcommand{\fx}{f(x)}

\newcommand{\gradfx}{\nabla \fx}
\newcommand{\Gx}{\nabla f(x)}
\newcommand{\Gk}{\nabla f(x_k)}
\newcommand{\Gs}{\nabla f(\xstar)}

\newcommand{\Hx}{\nabla^2 f(x)}
\newcommand{\Hk}{\nabla^2 f(x_k)}
\newcommand{\Hs}{\nabla^2 f(\xstar)}
\newcommand{\hess}{\nabla^2 f}

\newcommand{\step}{\alpha}
\newcommand{\Seqx}{\{ x_k \}}

\usepackage{mathtools}
\newcommand\myeq{\stackrel{\mathclap{\normalfont\mbox{def}}}{=}}

\usepackage{listings}
\lstset
{ 
    language=Matlab,
    basicstyle=\normalsize,
    numbers=left,
    stepnumber=1,
    showstringspaces=false,
    tabsize=1,
    breaklines=true,
    breakatwhitespace=false,
   frame=single
}


\begin{document}
\thispagestyle{empty}
\hrule
\begin{center}
   {\Large {\bf Programmazione 1 \hspace{3cm} $\quad \quad \quad$ Esercitazione 3}}
\end{center}
{\bf Cognome: }\hspace{2.5cm} {\bf Nome: } \hspace{2.5cm} {\bf Matricola: } \\\
\hrule

\section*{}
\begin{enumerate}

%%%%%%%%%%%%%%%%%%%%%%%%%%%%%%%%%%%%%%%%%%%%%%%%%%%%%%%%%%%%%%%%%%%%%%%%%%%%%
\item Mostrare come le due funzioni {\tt Sommatoria} e {\tt Produttoria} viste nell'esercitazione precedente, sono
entrambe dei casi particolare di una più generica funzione che possiamo chiamare {\tt Accumula} e specificare come segue:
\begin{python}
def Accumula(Operazione, ElementoNeutro, F, a, Next, b):
    # DA COMPLETARE COME ESERCIZIO
\end{python}
La funzione {\tt Accumula} prende in input tre funzioni:
\begin{enumerate}
\item {\tt Operazione}, che è la funzione che serve per "combinare" i valori calcolati
\item {\tt F}, che è la funzione da applicare a ciascun termine enumerato
\item {\tt Next}, che è la funzione che ci dice come calcolare il prossimo valore da valutare
\end{enumerate}		
		e prende in input 3 valori:
\begin{enumerate}
\item {\tt ElementoNeutro}, che è l'elemento neutro rispetto l'operazione definita nella procedura {\tt Operazione}
\item {\tt a}, il primo elemento della sequenza da considerare
\item {\tt b}, che è l'ultimo elemento da considerare 
\end{enumerate}

Si noti che a partire da questa funzione, si può definire la funzione {\tt Sommatoria}, come segue:
\begin{python}
def SommatoriaAcc(F, a, Next, b):
    return Accumulate(add, 0, F, a, Next, b)
\end{python}
Come possiamo definire la produttoria?

\begin{enumerate}
\item Scrivere una versione di accumulate che genera un processo ricorsivo. 
\item Scrivere una versione di accumulate che genera un processo iterativo. 
\end{enumerate}

%%%%%%%%%%%%%%%%%%%%%%%%%%%%%%%%%%%%%%%%%%%%%%%%%%%%%%%%%%%%%%%%%%%%%%%%%%%%%
\item Una funzione ancora più generale di {\tt Accumula} è la funzione {\tt FiltraAccumula} che introduce l'idea di 
avere un {\it filtro} sui valori da accumulare: vengono combinati dalla operazioni specificata in input solo quegli elementi
che soddisfano un certo predicato, descritto da una funzione {\tt Filter}. Implementare la seguente procedura:
\begin{python}
def FiltraAccumula(Filter, Operazione, ElementoNeutro, F, a, Next, b):
    # DA COMPLETARE COME ESERCIZIO
\end{python}
Usare la funzione definita sopra per calcolare:
\begin{enumerate}
\item La somma dei quadrati dei numeri primi nell'intervallo $[a,b]$.
\item Il prodotto di tutti i numeri primi minori di $n$ che sono primi rispetto ad $n$ (ovvero tutti i numeri interi positivi
$i < n$ tali che $MCD(n,i) = 1$.
\end{enumerate}

%%%%%%%%%%%%%%%%%%%%%%%%%%%%%%%%%%%%%%%%%%%%%%%%%%%%%%%%%%%%%%%%%%%%%%%%%%%%%
\item Scrivere un costruttore {\tt MakeRange(a, b)} che costruisce una lista (come catena di coppie, come visto a lezione)
di numeri interi compresi tra {\tt a} e {\tt b}.

\mybox{15}{1.5}

%%%%%%%%%%%%%%%%%%%%%%%%%%%%%%%%%%%%%%%%%%%%%%%%%%%%%%%%%%%%%%%%%%%%%%%%%%%%%
\item Scrivere una procedura {\tt Append(As, Bs} che prende in input due liste {\tt As} e {\tt Bs}, 
e restituisce un'unica lista che contiene prima tutti gli elementi della prima lista, e poi tutti quelli della seconda lista.

\mybox{15}{2.0}
 
%%%%%%%%%%%%%%%%%%%%%%%%%%%%%%%%%%%%%%%%%%%%%%%%%%%%%%%%%%%%%%%%%%%%%%%%%%%%%
\item Scrivere una procedura che prende in input una lista e restituisce la stessa lista ma con l'ordine degli elementi invertiti. 
Esempio: la lista (1, (2, (3,None))) diventa (3, (2, (1, None))).

\mybox{15}{2.5}

%%%%%%%%%%%%%%%%%%%%%%%%%%%%%%%%%%%%%%%%%%%%%%%%%%%%%%%%%%%%%%%%%%%%%%%%%%%%%
\item Scrivere una procedura {\tt Scale(Ls, a)} che moltiplica ogni elemento della lista {\tt Ls} per il termine {\tt a}.

\mybox{15}{1.5}

%%%%%%%%%%%%%%%%%%%%%%%%%%%%%%%%%%%%%%%%%%%%%%%%%%%%%%%%%%%%%%%%%%%%%%%%%%%%%
\item Scrivere una procedura {\tt Quadrati(Ls, a)} che calcola il quadrato di ogni elemento della lista {\tt Ls}.

\mybox{15}{1.5}

%%%%%%%%%%%%%%%%%%%%%%%%%%%%%%%%%%%%%%%%%%%%%%%%%%%%%%%%%%%%%%%%%%%%%%%%%%%%%
\item 
\begin{enumerate}
\item Scrivere una procedura {\tt Map(F, Ls)} che applica la funzione {\tt F} ad ogni elemento della lista {\tt Ls}.
\item Riscrivere le due funzioni precedenti {\tt Scale} e {\tt Quadrati}, usando la funzione {\tt Map}.
\end{enumerate}

\mybox{15}{2.5}

%%%%%%%%%%%%%%%%%%%%%%%%%%%%%%%%%%%%%%%%%%%%%%%%%%%%%%%%%%%%%%%%%%%%%%%%%%%%%
\item Scrivere una funzione {\tt Filter(P, Ls)} che prende in input un predicato {\tt P} e una lista {\tt Ls}
e restituisce in output la lista di elementi che soddisfano il predicato {\tt P}. Usare la funzione {\tt Filter}
e {\tt IsPrime} dell'esercitazione 2, per ottenere la lista dei numeri primi compresi tra 1 e 30.


\mybox{15}{2.5}
%%%%%%%%%%%%%%%%%%%%%%%%%%%%%%%%%%%%%%%%%%%%%%%%%%%%%%%%%%%%%%%%%%%%%%%%%%%%%
%\item {\bf CHALLENGE (facoltativo)}: Un autobotte che trasporta tre tipi di carburante da una raffineria ad un punto vendita
%ha una capienza di 41 Kl, suddivisi in 5 scomparti con capienza rispettivamente di 14, 5, 3, 7,e 11 Kl ciascuno. Ogni scomparto
%può contenere un solo tipo di carburante, ma carburanti dello stesso tipo di clienti diversi possono essere mischiati.

\end{enumerate}

\end{document}
