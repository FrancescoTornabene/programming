\documentclass[11pt,a4]{article}

\usepackage[margin=2cm]{geometry}

\usepackage{amsmath}
\usepackage{url}

\usepackage{amsmath}
\usepackage{url}

\usepackage[utf8]{inputenc}

% Default fixed font does not support bold face
\DeclareFixedFont{\ttb}{T1}{txtt}{bx}{n}{10} % for bold
\DeclareFixedFont{\ttm}{T1}{txtt}{m}{n}{10}  % for normal

% Custom colors
\usepackage{color}
\definecolor{deepblue}{rgb}{0,0,0.5}
\definecolor{deepred}{rgb}{0.6,0,0}
\definecolor{deepgreen}{rgb}{0,0.5,0}

\usepackage{listings}

% Python style for highlighting
\newcommand\pythonstyle{\lstset{
language=Python,
basicstyle=\ttm,
otherkeywords={self},             % Add keywords here
keywordstyle=\ttb\color{deepblue},
emph={MyClass,__init__},          % Custom highlighting
emphstyle=\ttb\color{deepred},    % Custom highlighting style
stringstyle=\color{deepgreen},
frame=tb,                         % Any extra options here
showstringspaces=false            % 
}}


% Python environment
\lstnewenvironment{python}[1][]
{
\pythonstyle
\lstset{#1}
}
{}

% Python for external files
\newcommand\pythonexternal[2][]{{
\pythonstyle
\lstinputlisting[#1]{#2}}}

% Python for inline
\newcommand\pythoninline[1]{{\pythonstyle\lstinline!#1!}}


\usepackage{collectbox}

\newcommand{\mybox}[2]{$\quad$\fbox{
\begin{minipage}{#1cm}
\hfill\vspace{#2cm}
\end{minipage}
}}


\usepackage{fancyhdr}
\pagestyle{fancy}
\rhead{Programmazione 1 - Esercitazione 4}

\usepackage[T1]{fontenc}
\usepackage[utf8]{inputenc}
\usepackage{lmodern}
%%%%%%%%%%%%%%%%%%%%%%%%%%%%%%%%%%%%%%%%%%%%%%%%%%%%%%%%%
% Source: http://en.wikibooks.org/wiki/LaTeX/Hyperlinks %
%%%%%%%%%%%%%%%%%%%%%%%%%%%%%%%%%%%%%%%%%%%%%%%%%%%%%%%%%
\usepackage{hyperref}
\usepackage{graphicx}
\usepackage[english]{babel}

\usepackage{bm}
\usepackage{amsmath}
\usepackage{amsfonts}

\usepackage{amsthm}
\newtheorem{definition}{Definizione}
\newtheorem{theorem}{Teorema}
\renewcommand*{\proofname}{Dimostrazione}
\newtheorem{example}{Esempio}
\newtheorem{lemma}{Lemma}
\newtheorem{exercise}{Esercizio}
\newtheorem{property}{Proprietà}

\usepackage[ruled,vlined,linesnumbered]{algorithm2e}

\newcommand{\xstar}{x^*}
\newcommand{\bxstar}{\bm{x^*}}
\newcommand{\bx}{\bm{x}}
\newcommand{\Rn}{\mathbb{R}^n}
\newcommand{\RR}{\mathbb{R}}
\newcommand{\norm}[1]{\left\lvert \left\lvert #1 \right\lvert \right\lvert}

\newcommand{\fx}{f(x)}

\newcommand{\gradfx}{\nabla \fx}
\newcommand{\Gx}{\nabla f(x)}
\newcommand{\Gk}{\nabla f(x_k)}
\newcommand{\Gs}{\nabla f(\xstar)}

\newcommand{\Hx}{\nabla^2 f(x)}
\newcommand{\Hk}{\nabla^2 f(x_k)}
\newcommand{\Hs}{\nabla^2 f(\xstar)}
\newcommand{\hess}{\nabla^2 f}

\newcommand{\step}{\alpha}
\newcommand{\Seqx}{\{ x_k \}}

\usepackage{mathtools}
\newcommand\myeq{\stackrel{\mathclap{\normalfont\mbox{def}}}{=}}

\usepackage{listings}
\lstset
{ 
    language=Matlab,
    basicstyle=\normalsize,
    numbers=left,
    stepnumber=1,
    showstringspaces=false,
    tabsize=1,
    breaklines=true,
    breakatwhitespace=false,
   frame=single
}


\begin{document}
\thispagestyle{empty}
\hrule
\begin{center}
   {\Large {\bf Programmazione 1 \hspace{3cm} $\quad \quad \quad$ Esercitazione 4}}
\end{center}
{\bf Cognome: }\hspace{2.5cm} {\bf Nome: } \hspace{2.5cm} {\bf Matricola: } \\\
\hrule

%%%%%%%%%%%%%%%%%%%%%%%%%%%%%%%%%%%%%%%%%%%%%%%%%%%%%%%%%%%%%%%%%%%%%%%%%%%%%
\section*{}

Obiettivo di questa esercitazione è di completare la vostra libreria {\tt pairslist} con tutte le funzionalità
più importanti di solito implementate in una struttura dati di tipo LISTA.

\begin{enumerate}


%%%%%%%%%%%%%%%%%%%%%%%%%%%%%%%%%%%%%%%%%%%%%%%%%%%%%%%%%%%%%%%%%%%%%%%%%%%%%
\item Scrivere un predicato {\tt Cointains(As, value)} che prende in input una lista {\tt As} e un valore {\tt value} 
e restituisce {\tt True} se la lista {\tt As} contiene il valore {\tt value}, {\tt False} altrimenti.

\mybox{15}{2.0}

%%%%%%%%%%%%%%%%%%%%%%%%%%%%%%%%%%%%%%%%%%%%%%%%%%%%%%%%%%%%%%%%%%%%%%%%%%%%%
\item Scrivere una procedura {\tt Count(As, value)} che prende in input una lista {\tt As} e un valore {\tt value} 
e conta il numero di volte che {\tt value} è contenuto in {\tt As}.

\mybox{15}{2.0}

%%%%%%%%%%%%%%%%%%%%%%%%%%%%%%%%%%%%%%%%%%%%%%%%%%%%%%%%%%%%%%%%%%%%%%%%%%%%%
\item Scrivere una procedura {\tt RemoveFirst(As, value)} che prende in input una lista {\tt As} e un valore {\tt value} 
e rimuove dalla lista il primo valore di {\tt value} nella lista {\tt As}.

\mybox{15}{2.0}

%%%%%%%%%%%%%%%%%%%%%%%%%%%%%%%%%%%%%%%%%%%%%%%%%%%%%%%%%%%%%%%%%%%%%%%%%%%%%
\item Scrivere una procedura {\tt RemoveAll(As, value)} che prende in input una lista {\tt As} e un valore {\tt value} 
e rimuove dalla lista tutti i valori di {\tt value} dalla lista {\tt As}.

\mybox{15}{2.0}


%%%%%%%%%%%%%%%%%%%%%%%%%%%%%%%%%%%%%%%%%%%%%%%%%%%%%%%%%%%%%%%%%%%%%%%%%%%%%
\item Si consideri la funzione {\tt FoldRight(P, As, v)} vista a lezione:

\begin{python}
def FoldRight(P, As, v):
    if IsEmpty(As):
        return v
    return P(Head(As), FoldRight(P, Tail(As), v))
\end{python}

Molti funzioni sulle liste possono essere riscritte in termini della {\tt FoldRight}.
Per esempio, la funzione {\tt Length(As)}, può essere scritta nel modo seguente:
\begin{python}
def FoldLength(Ls):
    """ Lunghezza di una lista in termini di fold """
    return FoldRight(lambda x,y: 1+y, Ls, 0)
\end{python}

Usando la funzione {\tt FoldRight}, si scrivano le funzioni seguenti:

\begin{enumerate}
\item {\tt Sum(As)}: calcola la somma dei valori nella lista {\tt As}

\mybox{15}{2.0}

\item {\tt Prod(As)}: calcola la produttoria dei valori nella lista {\tt As}

\mybox{15}{2.0}

\item {\tt Min(As)}: trova il minimo dei valori nella lista {\tt As}

\mybox{15}{2.0}

\item {\tt Max(As)}: trova il massimo dei valori nella lista {\tt As}

\mybox{15}{2.0}

\item {\tt FoldMap(F, As)}: definisce la funzione {\tt Map} in termini di {\tt FoldRight}

\mybox{15}{2.0}

\item {\tt FoldFilter(P, As)}: definisce la funzione {\tt Filter} in termini di {\tt FoldRight}

\mybox{15}{2.0}

\item {\tt FoldReverse(As)}: definisce la funzione {\tt Reverse} in termini di {\tt FoldRight}

\mybox{15}{2.0}

\end{enumerate}


\end{enumerate}

\end{document}
